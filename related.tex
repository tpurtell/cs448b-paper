%!TEX root = p.tex
\section{Related Work}

\cite{grobelnik2004visualization}
\cite{collins2006wordnet}
\cite{collins2009docuburst}
\cite{wermter2002selforganizing}
\cite{johnson2004network}
\cite{hearst1995tilebars}
\cite{mehler2006spatial}
\cite{paulovich2006text}



Sentiment analysis is an engaging capability to offer to users of a visualization.  Previous sucessful work has focused entirely on display sentiment analysis \cite{wanner2009visual}.  We included online sentiment analysis as a capability of system because it mapped directly onto features of our query engine.  Rather than focus on crafting sentiment word lists for the news analysis task, we used the General Inquirer database as a source of about a hundred sentiments \cite{generalinquirer}.

One of our original goals was to use the WordNet database as a means to provide guidance in term selection.  Visualization of the taxonomy of Enlgish words usually takes the form of a node link diagram \cite{dakka2008automatic,wordvis}.  Unfortunately, these presentations would consume too much space to be of practical use in our visualizations.  Intelligent auto-complete against related words would be more suitable for our purposes.


Should probably mention google insights/google trends