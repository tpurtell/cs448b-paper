%!TEX root = p.tex
\section{Discovery Interviews}

Over the course of development, we interviewed the following domain experts to gain insights and iterate on our solutions.
\begin{itemize*}
\item Christine - Journalism Ph.D. student
\item Geoff - Bill Lane Center for the American West at Stanford
\item Peter - Consultant and New York Times tech journalist
\item Ann - Director of the Graduate School of Journalism at Stanford
\item Justin - Current Knight Fellow, Formerly at the Washington Post
\item Phillip - In Charge of Data Visualizations at the Sacramento Bee
\end{itemize*}

Our goal for interviewing these users was to gain a preliminary understanding of how and why journalists/researchers query newspaper archives, as well as to gain a general understanding of the mindset and viewpoint of our users. 
We were interested in what sorts of questions journalists are trying to answer when they look through newspaper archives, and what tools and processes they employ.

Through these informal preliminary interviews a number of interesting insights emerged. Currently, most journalistic research occurs by querying for docuements directly through tools such as Lexis Nexis or Google. 
Journalists typically look through these archives when doing research on a person or event for a current story, and are often interested in researching what was said about a person or event in the past.
Querying for individual documents is a relatively solved problem through tools such as Lexis Nexis and Google, but no tools exist which can give journalists insights into aggregate trends across a large number of articles. 
The journalists we talked with view data as a collection of stories in hiding, so any interesting trends which can be discovered in old newspaper archives could themselves be the source of a story.

Since our dataset comprises 3 newspapers which are in decline, there was interest among the people we spoke with about being able to see what that decline means in terms of articles and topics. For instance, as newspaper staffs shrink, are more stories devoted to celebrity gossip rather than hard reporting? Are the topics of stories shifting? Is the sentiment in stories changing over time?

There was also interest from our users in interactive data visualization as the future of journalism. 
As newspapers are replaced with laptops and ipads, journalists are struggling to find interactive ways to tell stories rather than just using plain text. 
Currently, creating interactive visualizions online requires newspapers to pull together a team consisting of artists, data analysists, programmers, and journalists. 
Any tool which can make the process of finding stories in data, creating visualizations, and publishing easier is greatly needed by journalists, especially as staff sizes continue to shrink.
