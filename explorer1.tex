%!TEX root = p.tex
\section{Explorer   1}


\begin{figure}[htb]
  \centerline{
    \includegraphics[scale=0.25]{figures/explorer-prototype.png}
  }
  \caption{First iteration of Explorer interface}
  \label{fig:explorer-prototype}
\end{figure}

The first iteration of the ``Explorer" interface was designed to allow the visualization of the number of articles containing a given term, bucketed either by time or article page number. It borrowed the concept of ``series" from Excel and other familiar charting tools, but allowed the specification of series as unions of the sets of documents containing specified words. For instance, a series like ``dog OR cat" could be specified, and the series plotted would represent the number of documents containing one or both of these words. Additionally, this version allowed the user to specify a global filter, which would restrict the view to include only articles that match the series queries as well as the filter query. For instance, if one were to use ``pet" as the filter and ``dog" and ``cat" as series, the two plot lines would represent the result of running queries for ``pet AND dog" and ``pet AND cat." The version allowed the specification of arbitrary conjunctive normal form expressions for the filter. Beyond the series and filter features, this first iteration included a number of other features that are self-explanatory, including a year slider, numerous selection widgets for graph settings. 